\documentclass{article}
\usepackage[parfill]{parskip}
\usepackage{amsmath,amssymb,amsthm}
\usepackage{a4wide}
\usepackage[parfill]{parskip}
\usepackage{color}
\usepackage[utf8]{inputenc}
\usepackage{graphicx}
%\graphicspath{{./images/}}
\usepackage[english]{babel}
\usepackage{authblk}
\renewcommand{\epsilon}{\varepsilon}
\newcommand{\bigo}[1]{\mathcal{O}\left(#1\right)}
\newcommand{\note}[1]{\emph{\color{blue}#1}.\\}
\renewcommand{\L}{\mathcal{ L}}
\newcommand{\R}{\mathbb{ R}}

\title{Assignment Proposal: Numerical Homogenization}
\author[1]{Ivar Stefansson}
\author[2]{Omar Richardson}
\affil[1]{Department of Mathematics and Computer Science, Karlstad University}
\affil[2]{Department of Mathematics, University of Bergen} 

\begin{document}
\maketitle

\section{Project aim}
\label{sec:project_aim}

In this project, we want to compute effective parameters of a composite material numerically using a finite element simulation implemented in FeNiCS.
Our goal is to explore the relations between the finite element mesh size and the variation in material data.
We consider the Poisson equation with a rapid variation in the diffusion coefficient and explore different solution techniques, paying [mainly] attention to convergence rates and stability.

\section{Model}
\label{sec:model}
We consider a diffusion problem in a composite material. We denote the (periodic) domain with $\Omega \subset \R^d$, for $d\in\{1,2,3\}$. Within this material, we consider unknown quantity $u(x)$, of which the distribution is given by the solution of the Poisson equation:
\begin{equation}
    \begin{split}
        -\nabla \cdot (A_\epsilon(x)\nabla u(x)) &= f(x) \mbox{ for } x \in \Omega\\
        u(x) &= 0 \mbox{ for } x \in \partial\Omega
    \end{split}
    \label{eq:model}
\end{equation}

Here, $\epsilon$ is a small positive parameter, denoting the rapid variation of the material. 
For diffusivity $A_\epsilon(x)$, we choose a periodic function with a period of order $\epsilon$, like 
$$ A_\epsilon(x) = \left( 2+\cos(\frac{2\pi x}{\epsilon} \right)^{-1}.$$
For $\epsilon \to 0$, it is possible to obtain an homogenized diffusivity, called 'effective diffusion coefficient'.
If time permits, we will extend the model with a diffusion coefficient also varying in the normal length scale, and finally include time depence in the model as well.

\section{Methodology}
\label{sec:methodology}

We require some homogenization theory to analyse the effects of rapid variation on the solution of \eqref{eq:model}, and to compute effective diffusion coefficients. 
In addition, we need numerical linear algebra theory to explore different iterative solver options and apply the ones most efficient to our setting. 
Also, using results from asymptotic analysis, we are able to check our experimental convergence rates against theoretical upper bounds.

\section{Challenges}
\label{sec:challenges}

\section{Expected outcome}
\label{sec:expected_outcome}


\end{document}
