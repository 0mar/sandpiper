\documentclass{beamer}
\renewcommand\L{\mathcal{ L}}
\newcommand{\embed}{\hookrightarrow}
\newcommand{\embedembed}{\hookrightarrow\hookrightarrow}

\usetheme{Warsaw}
\usecolortheme{beaver}
\title[Numerical homogenization]{Numerical Homogenization in Fenics}
%\subtitle{An exploration}
\author[O. Richardson \and I. Stefansson] % (optional, for multiple authors)
{Omar Richardson \and Ivar Stefansson}
\institute % (optional)
{
    Karlstad University, Sweden \and University of Bergen, Norway
}
\date[]{Nordic Computational Course, 2017}
\subject{Numerical Homogenization}

\begin{document}
  \frame{\titlepage}
\begin{frame}
  \frametitle{Introduction}
  \begin{itemize}
    \item Composite Material
    \item Highly varying...
    \item Difficult to approximate
  \end{itemize}
\end{frame}

\begin{frame}[t]{Introduction}
  \frametitle{Project aim}
\end{frame}

\begin{frame}[t]{Theory}
  \frametitle{Upscaling}
\end{frame}

\begin{frame}[t]{Model}
  \frametitle{Model}
\end{frame}

\begin{frame}[t]{FEM approximation}
  \frametitle{Weak form}
\end{frame}

\begin{frame}[t]{FEM approximation}
  \frametitle{Exact solution}
\end{frame}

\begin{frame}[t]{FEM approximation}
  \frametitle{Comparison with numerical/upscaled}
\end{frame}

\begin{frame}[t]{FEM approximation}
  \frametitle{Convergence rates}
\end{frame}

\begin{frame}[t]{FEM approximation}
  \frametitle{2D}
\end{frame}

\begin{frame}[t]{FeNiCS acceleration}
  \frametitle{Solvers}
\end{frame}

\begin{frame}[t]{FeNiCS acceleration}
  \frametitle{Preconditioners}
\end{frame}

\begin{frame}[t]{Application}
  \frametitle{Application}
\end{frame}
\end{document}
